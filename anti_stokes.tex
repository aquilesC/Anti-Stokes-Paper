\documentclass[journal=nalefd,manuscript=letter]{achemso}
\setkeys{acs}{articletitle=true}
\usepackage{graphicx}
\usepackage{amsmath}
\usepackage{amssymb}
\usepackage{subfigure}
\usepackage{xr}
%%% This is for adding a footer with the time of the compilation of the file.
%%% Useful for versioning of printed copies.
\usepackage{datetime}
\usepackage{fancyhdr}
\fancyfoot[L]{\fontsize{8}{12} \selectfont \today $\,$ \currenttime}
\pagestyle{fancy}
%%%%%%%%%%%%%%%%%%%%%%%%%%%%%%%%%%%%%%%%%%%%%%%%%%%%%%%%%%%%%%%%%%%%%
%%%%%%%%%%%%%%%%%%%%%%%%%%%%%%%%%%%%%%%%%%%%%%%%%%%%%%%%%%%%%%%%%%%%%

\usepackage[version=3]{mhchem} % Formula subscripts using \ce{}
\usepackage[T1]{fontenc}       % Use modern font encodings
\externaldocument{supporting}

\newcommand{\K}{\ensuremath{\,\textrm{K}}}
\newcommand{\nm}{\ensuremath{\,\textrm{nm}}}
\newcommand{\mm}{\ensuremath{\,\textrm{mm}}}
\newcommand{\um}{\ensuremath{\,\mu\textrm{m}}}
\newcommand{\eV}{\ensuremath{\,\textrm{eV}}}
\newcommand{\uM}{\ensuremath{\,\mu\textrm{M}}}
\newcommand{\uW}{\ensuremath{\,\mu\textrm{W}}}
\newcommand{\pM}{\ensuremath{\,\textrm{pM}}}
\newcommand{\meV}{\ensuremath{\,\textrm{meV}}}
\newcommand{\pwr}{\ensuremath{\,\textrm{kW/cm}^2}}
\newcommand{\fs}{\ensuremath{\,\textrm{fs}}}
\newcommand{\ps}{\ensuremath{\,\textrm{ps}}}
\newcommand{\CPS}{\ensuremath{\,\textrm{CPS}}}
\newcommand{\kCPS}{\ensuremath{\,\textrm{kCPS}}}
\newcommand{\atto}{\ensuremath{\textrm{ATTO}\,647\textrm{N}}}


\author{Aquiles Carattino}
\affiliation[Leiden]
{Huygens-Kamerlingh Onnes Lab, 2300RA Leiden, The Netherlands}
\author{Michel Orrit}
\email{orrit@physics.leidenuniv.nl}
\affiliation[Leiden]
{Huygens-Kamerlingh Onnes Lab, 2300RA Leiden, The Netherlands}

\title{Gold nanoparticles as nano-thermometers}

\keywords{Gold nanorods, Plasmon, Anti-Stokes, Sensing, Temperature}

\begin{document}
\maketitle
\abstract{This is the abstract}
\section{Introduction}
Many physical, chemical and biochemical processes depend on temperature.
Together with the miniaturization of devices and the advent of nanotechnology
new methods for measuring temperature with high spacial accuracy started to
emerge. Notably the use of anti-Stokes fluorescence emission from single
molecules yield a high accuracy and localization. In biological processes,
measuring temperature with high accuracy has proven to be elusive, in part due
to methods requiring a careful sample preparation and calibration. 

New applications in photothermal therapy require locally increasing the
temperature in order to kill specific cells in a tissue. Many of this methods
employ metallic nanoparticles as heat sources but rely on models or in ad-hoc
calibrations to estimate the reached temperatures. Therefore a method that
allows both to increase the local temperature and to monitor it will be of great
interest in a broad range of fields. 

In recent years gold nanoparticles received a great amount of attention because
of their optical properties. A big cross section spanning through almost the entire
visible spectrum makes them easy to detect via exciting their luminescence, in
a dark field configuration or via photothermal imaging. Moreover their signal is
stable over time; gold nanoparticles do not blink nor bleach, making them ideal
candidates for biological labeling. 

Luminescence of metallic nanoparticles (and of gold particularly) has been
object of extensive debate in recent years. Different groups have tried to
quantitatively describe observed properties, as the quantum yield and the
emission spectrum. However the understanding of the processes involved is
fundamentally qualitative, mainly the interplay between the electron-hole pairs
and the plasmon resonance of the particles. 

The mechanism we propose in this work considers the radiative recombination of
electron and holes in the nanoparticle as the source of the luminescence. This
will be enhanced by the presence of the surface plasmon acting as an antenna.
The collective oscillation of conduction electrons plays a crucial role in the
interaction with photons, since it is responsible for the enhanced electric
field and absorption cross section. 

A monochromatic photon incident to the particle will give rise to a collective
oscillation of the gas of conduction electrons. The coherence of the oscillation
is broken due to collisions with the particle's surface and to the interaction
with baths. These can be mainly the electron-electron interactions,
electron-phonon coupling and the bath of photons coupled via spontaneous
emission. The decoherence time can be measured in pulsed experiments or
deduced from the inverse linewidth of the plasmon resonance and is in the order
of $6\fs$ or shorter. After this the state of the system can be described as a
superposition of hot electron and hole states. 

The hot electron and hole cool down by exchanging energy with the lattice on a
timescale of $\tau\approx1\ps$. Before this happens, electron and hole have a
small probability of recombining radiatively, re-emitting their high electronic
energy as a photoluminescence photon. If they have interacted only with static
surfaces, their energy would be the same and therefore the emitted photon would
have the same energy than the incoming photon, and will not contribute to the
measured photoluminescence (will be blocked by the notch filter.) If on the
other hand they have interacted with a phonon they have lost or acquired a
phonon energy and momentum quantum before recombining. 

Electron and hole can also interact with phonons upon recombination, either by
creation or annihilaton of a phonon. In both cases the energy available upon
recombination cannot exceed $\hbar\omega_L+k_BT$. It has to be noted that in the
hypothesis of a single-photon absorption, i.e. at low excitation power, the
temperature $T$ is that of the bath before absorption, the temperature of the
surrounding medium of the particle. This is different from pulsed experiments,
in which the electron gas temperature can be orders of magnitude higher than
room temperature. 

Radiative recombination gives rise to weakly emitting sources
spatially distributed throughout the particle and spectrally over a broad
frequency band with an exponential cutoff at $\hbar\omega_L+k_BT$. The
weak recombination emission can be greatly enhanced by the surface plasmon
resonance, acting as an antenna. With this model the following prediction can be
made. Firstly the emission spectrum must follow the plasmon spectrum if the
excitation laser is well above the plasmon resonance. If the excitation falls
within the plasmon resonance, the spectrum is expected to follow the plasmon
spectrum multiplied by Bose-Einstein statistics factor arising from phonon
population. This factor should be proportional to $\bar{n}$ for
anti-Stokes and $\bar{n}+1$ for Stokes processes, where

\begin{equation}
	\bar{n}=\left(\exp\frac{\hbar\Omega}{k_bT}-1\right)^{-1}.
\end{equation}

With this model, it can also be predicted that the emission should be polarized
as the plasmon; for gold nanorods this polarization coincides with the
longitudinal axis of the particle. Moreover, the lifetime should be determined
by the lifetime of hot electrons and holes and should be significatively shorter
than the thermalization time of the carriers. If this wasn't the case, few
interactions with phonons are enough to reduce the carriers' energy and
therefore the electron and hole wouldn't have enough energy to produce an
optical photon. Finally in the model only the presence of hot carriers is
required. As the wavevectors are randomly distributed at all times, the
recombination probability remains constant at all stages of relaxation.
Therefore excitation well above the plasmon resonance should excite the
photoluminescence with nearly the same efficiency as just above the plasmon
resonance. 

In this work we propose to use the anti-Stokes luminescence emission from gold
nanoparticles to determine the local temperature with nanometer accuracy. The
plasmon predicts an emission spectra with the shape,
\begin{equation}\label{eqn:fitting}
	I(\nu) =
	\textrm{SPR}(\nu)\cdot\left(\exp\frac{\textrm{h}(\nu-\nu_L)}{k_bT}-1\right)^{-1}
\end{equation}
where SPR is the surface plasmon resonance, $\nu_L$ is the frequency of the
laser, $h$ is Planck's constant, $k_b$ the Boltzmann constant and the only
remaining free parameter is the Temperature $T$ (plus a normalization constant
not included in \ref{eqn:fitting}.) This means that carefully fitting the
emission spectra allows to extract the local temperature without further
calibration.

\section{Experimental method}
All the measurements in this work were performed in a home-built confocal
microscope equipped with an APD and a spectrometer (Acton 500i). Samples were
mounted in a flowcell that allowed to increase the temperature of the medium up
to $60\,^oC$ and to monitor it through a Pt100 resistance thermometer placed
$1\mm$ away from the observation area.

Samples were prepared by spin casting a suspension of nanoparticles onto clean
coverslips. Different particles were employed in this work. Nanorods with 
average dimensions of $25\nm\times50\nm$ and a plasmon resonance around $630\nm$
and spheres with radii between $30\nm$ and $60\nm$. We employed a $532\nm$ (CNI)
laser for characterizing the nanorods' plasmon and for exciting spheres in
resonance. A $633\nm$ HeNe (Thorlabs) was employed to excite the nanorods in
resonance. 

Room temperature measurements were performed using an oil immersion objective
(Olympus) with a NA of $1.4$. This allowed a high excitation and collection
efficiency. For experiments at higher temperatures, a dry objective (Olympus)
was employed. This was needed to avoid the presence of a heat-sink close to the
observed area. The objective had a NA of $0.9$, making both the excitation and
collection efficiencies significantly lower. The first can be compensated
increasing the excitation power, the latter however is inherent to the method. 

To compensate for the drift of the setup while increasing the temperature, a
computer program was developed to continuously track a reference particle. The
same program was responsible for recording the temperature and triggering the
spectrometer. In this way complete data sets were acquired at different
temperatures, including spectra while exciting at $532\nm$, at $633\nm$ with
different laser intensities and the Pt100 value. 

We acquired several accumulations of the spectra at the same laser intensity.
This not only allows to lower the noise of the measurements (the same could
have been achieved by increasing the exposure time), it also allows to remove
bright pixels generated by cosmic rays and to monitor changes in the intensity
of the spectra, related to a drift of the setup while measuring. In case of a
recorded intensity change larger than acceptable, the particular spectra was
disregarded.

The estimation of the heating of the particles was performed with COMSOL, taking
into account not only the surrounding medium but also the supporting substrate.
The absorption cross section of the particles was calculated using Mie theory
for the case of spheres and the ADDA package for the case of nanororods. 

\section{Results}
As explained earlier, the proposed model for the anti-Stokes emission requires
to know the plasmon spectrum of the particle in order to fit the emission at
shorter wavelengths and extract its temperature. It has been shown that both
scattering and luminescence spectra overlap over a broad range of wavelengths.
Therefore exciting gold nanorods with $532\nm$ allows to record the longitudinal
plasmon spectra, as shown in the green solid curve of Figure
\ref{fig:spectra_rod}. The peak was fitted by a single lorentzian, shown in red
in the Figure; the dashed part of the curve depicts the spectral region that was
not considered for the fitting. It has to be reminded that the luminescence
spectra is not a perfect lorentzian since since there is a broadband
contribution to the lumienscence arising between the excitation wavelength and
the plasmon peak. This appears as an asymmetry in the emission spectrum,
particularly visible for wavelengths smaller than $625\nm$. The results of this
fitting will be employed for the SPR function defined in equation
\ref{eqn:fitting}. A lengthier discussion on the effects of this procedure will
be given later.

The red curves in Fig. \ref{fig:spectra_rod} show the Stokes emission of the
same nanorod at different excitation powers. With a different normalization
constant, they all show the same shape than the plasmon observed under $532\nm$
excitation. The blue curves in the Figure are the anti-Stokes emission at the
same excitation intensities than before. It can readily be seen that the shape
of this emission is exponential-like and doesn't follow the lorentzian shape of
the Stokes emission. The dip between Stokes and anti-Stokes is caused by the
notch filter that prevents direct excitation light from reaching the detectors.

The inset of Fig. \ref{fig:spectra_rod} shows the anti-Stokes-to-Stokes ratio of
the integrated luminescence for different laser excitation intensities. It is
possible to see that even with a linear behavior, the anti-Stokes intensity
increases more rapidly with laser excitation power than the Stokes emission.
This phenomenon was already exploited to image gold nanorods in high-background
conditions. Moreover it readily shows that the anti-Stokes emission depends on
laser excitation power differently than its Stokes counterpart. 

Figure \ref{fig:spectra_rod} shows the intensity of the Stokes (red) and
anti-Stokes (blue) emission for several excitation powers. In both cases the
linear fit in logarithmic scale has a slope close to $1$, being $0.9$ for the
Stokes and $1.2$ for the anti-Stokes, ensuring that both types of emission are
single-photon processes. The behavior is independent of the plasmon resonance
position. It is important to note that the excitation intensity cannot be
increased much beyond what is shown because nanorods would start reshaping into
spheres given enough energy.

Figure \ref{fig:ASS-ratio} shows the ratio of the anti-Stokes emission to the
Stokes emission for $100$ nanorods with different plasmon resonance and under
the same $633\nm$ excitation. The vertical red line depicts the laser
wavelength. It can be seen that the maximum ratio happens when the laser is
red-detuned from the resonance. This is the where the plasmon is
enhancing preferably the anti-Stokes emission. It has to be kept in mind that
off-resonance the cross section is lower and therefore the excitation is not as
efficient. Particles with a resonance at the laser wavelength in which nor the
anti-Stokes nor the Stokes is preferred, show a ratio close to $10\%$.

By fitting the spectra shown in \ref{fig:spectra_rod} with the equation
\ref{eqn:fitting} it is possible to extract the temperature of the particle at
each excitation power. Figure \ref{fig:AS_in_Log} shows the result of this
procedure. The spectra shown were recorded at $4$ different excitation
intensities; the full lines are the fitting results. The inset in the figure
shows the extracted temperatures at different intensities. Firstly is
possible to observe that the temperature is proportional to the excitation
intensity, and therefore to the absorbed energy as expected. From this results
it is possible to calculate the temperature at $0$ excitation power, i.e.
room temperature by extrapolating the results of the fit. The obtained value
in this is $293K$, while room temperature was set to $296K$. 

To prove that the temperature values obtained with this procedure are
reasonable, we performed finite element method simulations using Comsol.
Firstly We obtained the cross section from discrete dipole calculations
employing the ADDA package, using the average values for the dimensions of the
particles, and adjusting the length to obtain a plasmon overlapping the observed
one. Figure \ref{fig:simu_heating} shows the calculated temperatures under the
same excitation conditions. It can be seen a good agreement between both
measurement and simulations. Moreover it can be seen that once calculated the
cross section of the particle, the temperature of a rod or of a sphere with the
same volume are the same, therefore the temperature can be easily determined
from the absorbed power.

To study the dependence of the emission spectrum with the ambient conditions, we
proceeded to change the temperature of the surrounding medium. It has to be
remembered that in this set of experiments we employed a dry objective, and
therefore the excitation powers are larger to compensate for the lower
excitation efficiency but the collection efficiency cannot be improved. This
translates as fewer photons reaching the detector for the same laser intensity
at the sample plane. At each temperature several spectra were acquired at
different $633\nm$ excitation powers and also a $532\nm$ full spectrum in order
to monitor any possible reshaping of the particles during the experiment.

Figure \ref{fig:AS-temps-rods} shows the anti-Stokes emission spectra at the
same excitation power but at different temperatures. The changes in the slope of
the curves encodes the different temperatures. It is possible to note that there
is an increase in the noise of the spectra. It is possible to fit the more
intense curves with the same equation \ref{eqn:fitting} as before to extract the
temperature of the nanoparticle. The inset in Fig. \ref{fig:AS-temps-rods} shows
the extracted temperature from the fitting with the model as a function of the
temperature measured by the Pt100 in the vicinity of the nanorod. The red line
is showing the water temperature and acts as a guide to the eye. 

The inset of Figure \ref{fig:AS-temps-rods} clearly shows an increase in the
extracted temperature while increasing the temperature of the surrounding
medium. The range of explored temperatures went from $296\K$, room temperature,
up to $320\K$. This range is enough to observe a change in the anti-Stokes
emission spectrum. It has to be noted that the extracted temperature depends
also on the excitation power. At higher temperatures the stability of the setup
plays a crucial role in maintaining the particle in focus during the spectra
acquisition time. Longer exposure times and therefore lower excitation
intensities can be obtained if particles are actively maintained in focus.

Together with the excitation intensity, the plasmon position has a crucial role
in the accuracy of the extracted temperature. The luminescence spectrum acquired
with the $532\nm$ laser shows an asymmetric shape, due to the broadband
contribution of gold and the plasmonic emission. This makes the initial fitting
needed for the $SPR$ function of equation \ref{eqn:fitting} non univocally
determined. For the particle shown in Fig. \ref{fig:spectra_rod}, changing the
initial wavelength from $600\nm$ to $640\nm$ yields a significative difference
in the obtained parameters of the lorentzian fit.

Using the following shape for the SPR function in $\eV$, \begin{equation}*
SPR(E) = \frac{P_1}{(E-P_2)^2+P_3}
\end{equation}
The different fitting ranges give values for $P_2$ between $1.940\eV$ and
$1.955eV$, while the values of $P_3$ lie in between $5\cdot10^{-3}\eV^2$ and
$8\cdot10^{-3}\eV^2$. Figure \ref{fig:estimation-error}.a shows the extracted
temperature from the anti-Stokes spectra for all the possible parameter
combinations in the range just mentioned. Figure \ref{fig:estimation-error}.b
shows the temperature dependence while keeping either $P_2$ or $P_3$ constant.

In this example it is possible to note that the extracted temperature is highly
dependent on the fitted width of the plasmon spectrum but barely dependent on
the extracted peak position. From the initial spectrum of the particle, it is
possible to note that the wing of the plasmon coincides with the range of
anti-Stokes emission. Therefore minute changes in the initial shape will yield
higher changes in the extracted temperature.

This assertion implies that there should be particles in which the extracted
temperature wouldn't be too sensitive to the initial plasmon fit. To explore
these possibilities, we calculated the anti-Stokes emission of different
particles at $400\K$ by using equation \ref{eqn:fitting} and the results of ADDA
calculations for the $SPR$ function. We then extracted a temperature from these
spectra while varying the lorentzian parameters as was done for the experimental
results. In this way it is possible to study the expected error in temperature
generated by the initial uncertainty in the fitting of the plasmon for different
particles. 

Figure \ref{fig:calculated-error} shows the uncertainty in the extracted
temperature as a function of the plasmon peak position. The uncertainty is
defined as the difference of the maximum and the minimum extracted temperatures
while varying $P_2$ by $10\%$ and $P_3$ by $30\%$. The vertical red line depicts
the position of the laser.



% It was not possible, however, to extrapolate to zero excitation power, since
% only few spectra were clear enough as to proceed with the fitting.

To increase the power employed and compensate for the lower collection
efficiency, it is possible to employ gold nanospheres instead of nanorods. The
main practical difference is that they can withstand much higher excitation
powers without reshaping. Moreover the plasmon resonance of spheres shifts only
slightly with radius, therefore is is possible to predict it using
Mie theory and eliminating the need of a second laser beam. 

An example of the spectrum of a $60\nm$ sphere under $532\nm$ irradiation at
different excitation powers can be seen in Figure \ref{fig:60sphere}.
As in the case of gold nanorods, two distinct parts of the spectrum can be
observed; in red the Stokes emission and in blue the anti-Stokes emission. The
linearity of both is presented on the left inset of the figure. It has to be
noted however that all spheres will have approximately the same resonance and
therefore it will not be possible to find one in which the anti-Stokes emission
is more favorable, as was done for the rods. In this case using a laser with a
different wavelength would have been advisable. 

It has to be noted that spheres have a quantum yield one order of magnitude
lower than rods; the same is true for the anti-Stokes emission. Moreover, the
luminescence intensity is proportional to the volume, thus scaling as the third 
power of the radius. It means that going from $60\nm$ radius to $30\nm$ will
diminish the luminescence intensity by a factor $8$. The spectra of a $40\nm$
and a $30\nm$ radius spheres is shown in Figure \ref{fig:30-40}. It is still
possible to observe the anti-Stokes emission but the noise in the spectra is
much more significative than before. 

We performed the same measurements than with rods, varying the temperature of
the surrounding medium. Figure \ref{fig:spheres-temperature} shows the extracted
temperature for different measured temperatures for spheres of different radii.
It is remarkable that the three spheres follow the same trend. The inset of
the Figure shows the anti-Stokes emission of the $40\nm$ sphere for different
temperatures and their fitting with the model of equation $\ref{eqn:fitting}$.





\end{document}
