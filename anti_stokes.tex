\documentclass[twocolumn]{article}
\usepackage{graphicx}
\usepackage{amsmath}
\usepackage{amssymb}
\usepackage{subfigure}
\usepackage{epstopdf}
\usepackage{setspace}
\onehalfspacing
\epstopdfsetup{update} % only regenerate pdf files when eps file is newer
\title{Single Gold Nanorods as nano thermometers}
\author{Aquiles Carattino \and Michel Orrit}

\begin{document}
\maketitle
\abstract{This is the abstract}

\section{Introduction}
In recent years gold nanoparticles (with different shapes) received a big
amount of attention because of their optical properties. A big cross section
spanning through almost the entire visible spectrums makes them easy to detect
via exciting their luminescence, in a dark field configuration or via
photothermal imaging. Moreover their signal is stable in time, gold
nanoparticles do not blink nor bleach, making them ideal candidates for
biological labeling. 

One of the properties that make gold nanoparticles interesting is the ability to
tune the resonance wavelength by changing their shape (aspect ratio in the case
of rods or spheres). This resonance is called the plasmon resonance and it can
be found at around $540$nm for spheres (aspect ratio of 1) until the near
infrared region. The resonance peak can be accurately described by simple models
and can be predicted with numerical software. It has to be remembered that is
not only the shape that determines its position but also the refractive index of
the medium in which they are immersed. 

Another interesting aspect of gold nanorods is their ability to dissipate heat
to a localized environment. Their high absorption cross section but low quantum
yield means that most of the energy is dissipated as heat. This fenomenon is
exploited in different ways, for instance photothermal heterodyne imaging is
able to detect particles down to $5$nm in radius; nanobubbles can be generated
by exciting with enough power. The same principles are also used for
photothermal therapy since the ability to target specific cells will provide
ways of increase the temperature and kill specific types. Also the shockwave
generated by the bubble formation can be used to target specific regions in a
tissue, for instance. 

For all the methods mentioned before, it is of utmost importance to monitor the
temperature that is reached for different laser powers or at different regions
of the sample. There are, however, no simple methods for achieving this. Many
rely on the energy transfer between two different dyes and a spectral analysis
of the emission. This method can achieve a high degree of accuracy, but has to
be carefully calibrated and the sample preparation, specially if with biological
implications, is far from simple. 

All the before mentioned methods have been proved to be useful and successful
but so far they are lacking a simple and effective way of monitoring the
temperature they are reaching while varying the laser power. This work focus on
the anti-Stokes luminescence from gold nanorods as a way for doing so in a
self-calibrated, burden-free manner. 

Most works in the past have focus in explaining the Stokes-shifted luminescence
from rods and there is still some room for debate. In a general way, it can be 

\section{Experimental method}
The main advantage of the method proposed in this work is that any confocal
microscope equipped with a spectrometer is ready to perform the measurement. In
the present case, gold nanorods were spincoated into glass coverslips and rinsed
with milli-Q water to remove any excess of CTAB present. A simple $532$nm diode
laser was employed to acquire the full emission spectrum of individual
particles. A HeNe (Thorlabs ..) with a wavlength of $632.8$nm (from now own
referred as $633$nm) is employed to excite the particles close to the plasmon
resonance. An AOM placed in its path allows to vary the intensity in a fast and
repetitive manner.

Two different objectives are employed for the mesurements at room or at higher
temperature. In the first case a high NA oil-immersion objective was employed
(Olympus ..), while for the latter an Olympus 60X NA 0.9 is chosen to minimize
the thermal contact between the sample and the optics. The samples were placed
in a flowcell in which the temperature can be regulated by flowing warm or cold
water. A thermo resistance (PT$100$) mounted close to the observation position
allows to monitor the local temperature in an independent way.

The output of the objective was filter via a confocal pinhole and sent to an APD
(Perkin-Elmers) for recording images and to a Spectrometer (Acton 500-i) through
a non polarizing beamsplitter cube. The full schematic of the setup can be found
in the supporting information. For filtering the excitation light, notch filters
are uses for the $532$nm laser and a combination of notch (Semrock \ldots) and
longpass (Semrock \ldots) or shortpass (Semrock \ldots) for the $633$nm.

\section{Results}
Figure \ref{fig:spectra1} shows the three key elements of this work. The green
curve with squares shows the emission spectra of a particle while being
irradiated with $532$nm laser. The cut at around this wavelength is due to the
notch filters employed. The peak at around $650$nm is normally referred as the
longitudinal plasmon. The curve in red with open circles shows the emission
while the same particle is irradiated with $633$nm laser and a longpass filter
is used to cut out the excitation light. It was normalized a $650$nm to show the
overlap with the $532$nm curve. This emission will be referred as the Stokes
emission from now on. The third curve (blue, full circles) shows the emission of
the gold nanorod to shorter wavelengths (higher energies) when a shortpass
filter is employed. This portion of the spectra will be referred to as the
anti-Stokes emission and is the one in which we are going to focus, since it
encodes the temperature information. 

One of the first things to realize is that the anti-Stokes luminescence depends
heavily with the plasmon peak position. On one hand the absorption cross section
of a particle coincides with its plasmon therefore exciting at the resonance
frequency yields the highest absorption. However, as can be seen in Figure
\ref{fig:intensity_plasmon_position}.a, the maximum anti-Stokes emission happens
for particles with the resonance peak slightly shifted to the blue. This is
result is however not surprising if we consider the presence of the filters, for
which the Stokes emission (Figure \ref{fig:intensity_plasmon_position}.b) may
give a better picture. Exciting in resonance also means loosing the
resonance peak since its going to be blocked by the filters. If the peak is
slightly shifted, the excitation may be slightly smaller, but the collection
efficiency will be larger. 

Polarization analysis of both the emission and the excitation yield a perfect
matching between Stokes and anti-Stokes. When the excitation laser is parallel
to the longitudinal axes of the nanorod, the cross section is maximized and
therefore the emission is the highest for both cases. On the other hand, the
Stokes-shifted emission has long be known to be polarized along the axis of the
particle. This same behavior is observed for the anti-Stokes. Combinig this two
simple observations makes clear that there will be a plasmonic component in the
anti-Stokes emission and therefore cannot be neglected. 

Figure \ref{fig:Intensity_excitation} shows the dependence of both the Stokes
and the anti-Stokes emission with the excitation power. To a first degree it is
possible to observe the linear increase of signal meaning that this process is a
one-photon excitation of the particles. This has already been discussed and
investigated in previous works, both in bulk and at the single-particle level. 

The statement of the previous paragraph holds only for a first order
approximation of the curves. Figure \ref{fig:several_powers_room_temperature}
shows different anti-Stokes emission spectra for different excitation powers.
The logarithmic scale in the ordinates allows to see that changing the intensity
of the laser not only increases the emission intensity but also changes the
slope of it. Stricktly speaking this means that doubling the excitation power
does not double the emission intensity, but that there is a non linear effect
that will be shown later is due to an increase in temperature of the particle
and its surrounding. 

By fitting the spectra shown in \ref{fig:several_powers_room_temperature} with
the equation \ref{eqn:fitting_anti_stokes} it is possible to extract the surface
temperature of the particle at each excitation power. The first step is to
extract the lorentzian shape of the plasmon peak by fitting the curve obtained
with the $532nm$ excitation wavelength. Then, the extracted temperatures are
shown in Figure \ref{fig:temp_vs_power_room_temperature}; the errorbars are
derived from the uncertainty in the slope in the linear fitting of the spectra.
This readly shows that it is possible to measure the particle's temperature with
an accuracy of 2\% for higher powers and with 5\% for lower ones. It can be seen
that this increase in uncertainty comes from a lower intensity being recorded by
the spectrometer, therefore increasing the exposure time may improve the
results. The other source of uncertainty comes from the fitting of the plasmon
peak. It can be seen that by shifting the value of the plasmon peak position the
temperature extracted changes by several degrees. It is therefore of outmost
importance to acquire the spectra while exciting at $532$nm with the best
accuracy. 

The results discussed above can be analyzed even further. A linear fitting of
the data shown in Figure \ref{several_powers_room_temperature} allows to extrapolate
the data to zero excitation power; this is equivalent to measure room
temperature. In this case the value extracted was $111$K that coincides
precisely with the temperature in the laboratory at the time of the measurement.
To show the utility of this technique to extract the temperature of the
surrounding medium, we performed the same experiment while varying the
temperature of the environment in which the particles are immersed. 

Because of the design of the flowcell and to improve the thermal conductivity,
the rods are placed in immersion oil. This will shift the plasmon resonance to
the red as compared to water, but nevertheless it is possible to find particles
with its resonance close enough to our laser. Figure \ref{fig:Temp_Vs_Temp}
shows the extrapolated zero-excitation temperature as a function of the
temperature in the flowcell. 
%AGREEMENT OR NOT? 


\end{document}
